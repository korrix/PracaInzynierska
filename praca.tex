\documentclass[12pt]{aghdpl}
\usepackage[english,polish]{babel}
\usepackage{polski}
\usepackage[utf8]{inputenc}

\usepackage{mathtools}
\usepackage{amsfonts}
\usepackage{amsmath}
\usepackage{amsthm}
\usepackage{bussproofs}
\usepackage{mdframed}
\usepackage[final]{microtype}

\usepackage{tikz}
\usetikzlibrary{shapes.geometric, arrows, shapes.multipart, positioning}
% --- < bibliografia > ---

\usepackage[
style=numeric,
sorting=none,
block=space,
%
% Zastosuj styl wpisu bibliograficznego właściwy językowi publikacji.
language=autobib,
autolang=other,
% Zapisuj datę dostępu do strony WWW w formacie RRRR-MM-DD.
urldate=iso8601,
% Nie dodawaj numerów stron, na których występuje cytowanie.
backref=false,
% Podawaj ISBN.
isbn=true,
% Nie podawaj URL-i, o ile nie jest to konieczne.
url=false,
%
% Ustawienia związane z polskimi normami dla bibliografii.
maxbibnames=3,
% Jeżeli używamy BibTeXa:
backend=biber
]{biblatex}

\usepackage{csquotes}
% Ponieważ `csquotes` nie posiada polskiego stylu, można skorzystać z mocno zbliżonego stylu chorwackiego.
\DeclareQuoteAlias{croatian}{polish}

\addbibresource{bibliografia.bib}

% Nie wyświetlaj wybranych pól.
%\AtEveryBibitem{\clearfield{note}}


% ------------------------
% --- < listingi > ---

% Użyj czcionki kroju Courier.
\usepackage{courier}

\usepackage{listings}
\lstloadlanguages{Haskell}
\definecolor{mygray}{gray}{0.5}
\lstset{
  frame=shadowbox,
  aboveskip=\parindent,
  belowskip=\parindent,
  belowcaptionskip=-1em,
  numbers=left,
  numberstyle=\scriptsize\ttfamily\color{mygray},
  rulesepcolor=\color{mygray},
  stepnumber=1,
  showstringspaces=false,
  framesep=5pt,
  language=Haskell,
  captionpos=b,
	literate={ą}{{\k{a}}}1
           {ć}{{\'c}}1
           {ę}{{\k{e}}}1
           {ó}{{\'o}}1
           {ń}{{\'n}}1
           {ł}{{\l{}}}1
           {ś}{{\'s}}1
           {ź}{{\'z}}1
           {ż}{{\.z}}1
           {Ą}{{\k{A}}}1
           {Ć}{{\'C}}1
           {Ę}{{\k{E}}}1
           {Ó}{{\'O}}1
           {Ń}{{\'N}}1
           {Ł}{{\L{}}}1
           {Ś}{{\'S}}1
           {Ź}{{\'Z}}1
           {Ż}{{\.Z}}1,
	basicstyle=\footnotesize\ttfamily,
}

% ------------------------

\AtBeginDocument{
	\renewcommand{\tablename}{Tabela}
	\renewcommand{\figurename}{Rys.}
}

% ------------------------
% --- < tabele > ---

\usepackage{array}
\usepackage{tabularx}
\usepackage{multirow}
\usepackage{booktabs}
\usepackage{makecell}
\usepackage[flushleft]{threeparttable}

% defines the X column to use m (\parbox[c]) instead of p (`parbox[t]`)
\newcolumntype{C}[1]{>{\hsize=#1\hsize\centering\arraybackslash}X}


%---------------------------------------------------------------------------

\author{Konrad Lewandowski}
\shortauthor{K. Lewandowski}

\titlePL{Obliczenia rozproszone w~języku Haskell}
\titleEN{Distributing tasks with Haskell}

\shorttitlePL{Obliczenia rozproszone w~języku Haskell}
\shorttitleEN{Distributing tasks with Haskell}

\thesistype{Praca dyplomowa inżynierska}
\supervisor{dr inż. Piotr Matyasik}

\degreeprogramme{Informatyka}

\date{2018}

\department{Katedra Informatyki Stosowanej}

\faculty{Wydział Elektrotechniki, Automatyki,\protect\\[-1mm] Informatyki i Inżynierii Biomedycznej}

\acknowledgements{Serdecznie dziękuję \dots tu ciąg dalszych podziękowań np. dla promotora, żony, sąsiada itp.}


\setlength{\cftsecnumwidth}{10mm}

%---------------------------------------------------------------------------
\setcounter{secnumdepth}{4}
\brokenpenalty=10000\relax

\begin{document}

\titlepages

% Ponowne zdefiniowanie stylu `plain`, aby usunąć numer strony z pierwszej strony spisu treści i poszczególnych rozdziałów.
\fancypagestyle{plain}
{
	% Usuń nagłówek i stopkę
	\fancyhf{}
	% Usuń linie.
	\renewcommand{\headrulewidth}{0pt}
	\renewcommand{\footrulewidth}{0pt}
}

\setcounter{tocdepth}{2}
\tableofcontents
\newpage
\renewcommand\lstlistlistingname{Listingi kodu}
\lstlistoflistings
\clearpage
% !TEX root = praca.tex

\chapter{Wprowadzenie}
\label{cha:wprowadzenie}
W czasach kiedy przewidywania Gordona Moore'a dotyczące dalszego wzrostu mocy obliczeniowej pojedynczych komputerów przestają się sprawdzać, coraz częściej wykorzystujemy metody wykonywania programów oparte na jednoczesnym przetwarzaniu rozproszonym na wielu połączonych ze sobą maszynach. 

Pomimo pozornej prostoty takiego rozwiązania istnieje bardzo niewiele narzędzi ułatwiających programowanie w modelu rozproszonym.

%---------------------------------------------------------------------------
\section{Cel i założenia pracy}
\label{sec:celePracy}
Celem poniższej pracy jest implementacja w języku Haskell podstawowej biblioteki do rozpraszania zadań na wielu komputerach z wykorzystaniem brokera RabbitMQ, umożliwiającą zlecanie zadań do wykonania, przerywanie zadań będących w trakcie wykonywania, raportowanie na bieżąco postępów wykonania, przesyłania wyników oraz wprowadzanie zależności pomiędzy zadaniami.

%---------------------------------------------------------------------------
\section{Zawartość pracy}
\label{sec:zawartoscPracy}
W rozdziale 1 cośtamcośtam...


















% !TEX root = praca.tex

\chapter{Implementacja rozpraszania zadań}
Poniższy rozdział został poświęcony różnym aspektom implementacyjnym.

\section{Zarządzanie zasobami}
Protokół AMQP operuje na zależnych od siebie zasobach, którymi należy w poprawny sposób zarządzać. Postawowy przykład z dokumentacji biblioteki AMQP w przypadku wystąpienia wyjątku nie gwarantuje zwolnienia zasobu, a kolejność wykonywania finalizatorów ze względu na leniwą ewaluację jest determinowana wyłącznie wyzwoleniem mechanizmu odśmiecania.

\begin{lstlisting}[caption=Łączenie z RabbitMQ]
main = do
  conn <- openConnection "127.0.0.1" "/" "guest" "guest"
  ...
  closeConnection conn
\end{lstlisting}

Jednym z istniejących rozwiązań tego problemu jest biblioteka \texttt{io-region}\cite{IoReg}, umożliwiająca podział kodu na regiony odpowiedzialne za poszczególne zasoby oraz przenoszenie tych odpowiedzialności pomiędzy regionami

\begin{lstlisting}[caption=Regionalizacja zasobów]
...
region $ \r -> do
 connection <- alloc_ r (AMQP.openConnection host vhost username password)
                         AMQP.closeConnection
  -- po opuszczeniu regionu następuje zwolnienie zasobów
...
\end{lstlisting}

Rejestrowanie zasobów w obrębie odpowiednich regionów rozwiązuje również problem asynchronicznych wywołań zwrotnych, służących do obsługi nadchodzących komunikatów z brokera. Dla porównania, niewłaściwe rozwiązanie oparte o mechanizm \lstinline{bracket} powoduje przedwczesne zamknięcie zasobu nadal wykorzystywanego przez funkcję uruchamianą asynchronicznie.
\newpage
\begin{lstlisting}[caption=Problem funkcji asynchronicznych]
...
bracket openConnection closeConnection $ \connection ->
  bracket (openChannel connection) closeChannel $ \channel1 ->
    consumeMsgs channel1 queue callback -- wywołanie asynchroniczne
  -- nieporządane zamknięcie kanału
  -- otwarcie drugiego kanału
  bracket (openChannel connection) closeChannel $ \channel2 ->
    consumeMsgs channel2 queue callback -- wywołanie asynchroniczne
    -- nieporządane zamknięcie kanału
  _ <- getLine -- oczekiwanie
-- zamknięcie połączenia

...
region $ \r -> do
  connection <- alloc_ r openConnection closeConnection
  channel1 <- alloc_ r (openChannel connection) closeChannel
  consumeMsgs channel1 queue callback
  channel2 <- alloc_ r (openChannel connection) closeChannel
  onsumeMsgs channel2 queue callback

  _ <- getLine -- oczekiwanie
  -- zwolnienie zasobów w poprawnej kolejności
\end{lstlisting}
\newpage

\section{Środowisko funkcji}
Większość funkcji związanych z protokołem AMQP wymaga do działania przekazania pewnego zasobu (jak na przykład obiekt \lstinline{Connection} lub \lstinline{Channel}). Robienie tego za każdym razem explicite prowadzi do zmniejszenia czytelności kodu. Idiomatycznym dla języka Haskell rozwiązaniem jest zastosowanie typu \lstinline{Reader} \cite{Reader}: 
\begin{lstlisting}[caption=Typ reader]
newtype Reader e a = Reader { runReader :: e -> a }

instance Functor (Reader e) where
  fmap f r = Reader $ \e -> f (runReader r e)

instance Applicative (Reader e) where
  pure a    = Reader $ \e -> a
  ra <*> rb = Reader $ \e -> (runReader ra e) (runReader rb e)

instance Monad (Reader e) where 
  (Reader r) >>= f = Reader $ \e -> runReader (f (r e)) e

ask :: Reader a a
ask = Reader id
\end{lstlisting}

Typ \text{Reader} opakowuje funkcję przyjmującą jako argument środowisko jej wykonania i zwracającej pewien wynik obliczeń wykorzystujących to środowisko. Przykładowo dla:

\begin{lstlisting}
testReader :: Reader Bool String
testReader = Reader $ \flag -> if flag then "Włącz" else "Wyłącz"

runReader testReader True == "Włącz"
\end{lstlisting}

Jednak istotą działania \lstinline{Reader}'a jest jego monadyczny interfejs umożliwiający zapisanie funkcji \lstinline{testReader} jako:

\begin{lstlisting}
testReader :: Reader Bool String
testReader = do
  flag <- ask
  return $ if flag then "Włącz" else "Wyłącz"
\end{lstlisting}

Dzięki temu unikamy przekazywania tego samego środowiska za każdym razem jako argumentu funkcji, co jest szczególnie istotne w przypadku kiedy jest to wiele funkcji.

Niestety funkcjonalność samej monady \lstinline{Reader} reader nie jest wystarczająca ze względu na mnogość operacji wykorzystujących operacje wejścia-wyjścia, więc wymagających typu \lstinline{IO} do działania. Pomocne tutaj okazują się transformatory monad (ang. \textit{Monad transformers}\cite{Transformers}):

\begin{lstlisting}[caption=Transformator typu Reader]
newtype ReaderT e m a = ReaderT { runReaderT :: e -> m a }

instance Functor m => Functor (ReaderT e m) where
  fmap f r = ReaderT $ \e -> fmap f (runReaderT r e)

instance Applicative m => Applicative (ReaderT e m) where
  pure a    = ReaderT $ \e -> pure a
  ra <*> rb = ReaderT $ \e -> (runReaderT ra e) <*> (runReaderT rb e)

instance Monad m => Monad (ReaderT e m) where 
  r >>= f = ReaderT $ \e -> do
    a <- runReaderT r e
    runReaderT (f a) e

ask :: Monad m => ReaderT a m a
ask = ReaderT return

lift :: m a -> ReaderT e m a
lift m = ReaderT (const m)
\end{lstlisting}
Dzięki funkcji \lstinline{lift} możemy niejako ,,podciągać'' operacje wykonane w ramach innej monady do typu \lstinline{ReaderT}:

\begin{lstlisting}
testReader :: ReaderT Bool IO String
testReader = do
  flag <- ask 
  lift $ if flag then putStrLn "Włącz"
                 else putStrLn "Wyłącz"
  return "Wynik"

> runReaderT testReader True
Włącz
\end{lstlisting}

\newpage
\section{Odczyt konfiguracji z pliku}
Biblioteka \texttt{configurator}\cite{Config} TODO

\section{Obsługa zadań}
Mechanizm uruchamiania konkretnego zadania sprowadza się do odczytu odpowiedniej funkcji z tablicy mieszającej zawierającej wszystkie obsługiwane przez węzeł zadania na podstawie klucza będącego wybraną przez programistę nazwą zadania, a następnie uruchomienie (po odpowiednim sparametryzowaniu) tej funkcji wewnątrz oddzielnego wątku. Ogólny schemat wygląda następująco:

\begin{lstlisting}[caption=Schemat obsługi zadania]
newtype Task = Task { runTask :: ByteString -> IO ByteString }

registeredTasks :: HashMap Text Task
registeredTasks = fromList [ ("name1", function1)
                           , ("name2", function2)
                           , ("name3", function3) ]
...

handleTask (msg, env) = do
  let Just taskName = AMQP.msgType msg
  let      taskArgs = AMQP.msgBody msg

  forkIO $
    result <- runTask (registeredTasks ! taskName) taskArgs
    ...
  
  AMQP.ackEnv env
\end{lstlisting}
Nieprzypadkowo typ \lstinline{Task} to opakowana, monomorficzna funkcja przetwarzająca ciągi bajtów. Gdyby pokusić się o przeniesienie odpowiedzialności za deserializację argumentów i serializację wyników obliczeń, typ ten przyjąłby pozornie ,,bezpieczniejszą'' formę egzystencjalną:
{\large $$\forall_{a, b} (\mathrm{Serializable}\ a, \mathrm{Serializable}\ b) \Rightarrow a \rightarrow b$$}%
jednak w takiej sytuacji kompilator nie może ustalić typów polimorficznych $a$, $b$ w kontekście fragmentu kodu obliczającego wartość:
\newpage
\begin{lstlisting}[caption=Problem typu zadania zdefiniowanego egzystencjalnie]
{-# LANGUAGE ExistentialQuantification #-}
{-# LANGUAGE RankNTypes                #-}
module Test where

import Data.Store
import Data.Text
import Data.ByteString
import Data.HashMap.Strict

newtype Task = Task { runTask :: forall a b . (Store a, Store b) 
                              => a -> IO b }

registeredTasks :: HashMap Text Task
registeredTasks = ...

runTaskByName :: Text -> ByteString -> IO ByteString
runTaskByName taskName encodedArg = do
  arg <- decodeIO encodedArg
  result <- runTask (registeredTasks ! taskName) arg
  -- Błąd sprawdzania jednoznaczności typów
  return $ encode result
  
\end{lstlisting}

Błąd ten jest analogiczny do przedstawionego poniżej bardziej podstawowego przykładu i wychwytuje go mechanizm sprawdzania jednoznaczności instancjonowanych typów:
\begin{lstlisting}
show (read "5" :: Int)         => "5"
show (read "5" :: Double)      => "5.0"
show (read "5") => Ambiguous type variable `a2' arising from a use of `read'
                   prevents the constraint `(Read a2)' from being solved.
\end{lstlisting}

Inferencja typu dla trzeciego wyrażenia przebiega w systemie Hindleya-Milnera\cite{HM} rozszerzonym o klasy typów\cite{TC} następująco:
\begin{figure}[h]
\scriptsize 
\begin{mdframed}
\begin{prooftree}
  \AxiomC{$\mathrm{show} : \forall \alpha.\ (\mathrm{Show}\ \alpha).\ \alpha \rightarrow \mathrm{String} \in \Gamma$}
  \RightLabel{$\left[\text{Var}\right]$}
  \UnaryInfC{$\Gamma \vdash \mathrm{show} : \forall \alpha.\ (\mathrm{Show}\ \alpha).\ \alpha \rightarrow \mathrm{String}$}
 
  \AxiomC{$\mathrm{\langle 5\rangle} : \mathrm{String} \in \Gamma$}
  \RightLabel{$[\text{Var}]$}
  \UnaryInfC{$\Gamma \vdash \mathrm{\langle 5\rangle} : \mathrm{String}$}
  
  \AxiomC{$\mathrm{read} : \forall \alpha.\ (\mathrm{Read}\ \alpha).\ \mathrm{String} \rightarrow \alpha \in \Gamma$}
  \RightLabel{$\left[\text{Var}\right]$}
  \UnaryInfC{$\Gamma \vdash \mathrm{read} : \forall \alpha.\ (\mathrm{Read}\ \alpha).\ \mathrm{String} \rightarrow \alpha$}
  
  \RightLabel{$[\text{App}]$}
  \BinaryInfC{$\mathrm{read\ \langle 5\rangle} : \forall \alpha.\ (\mathrm{Read}\ \alpha). \alpha $}

  \RightLabel{$[\text{Comb}]$}
  \BinaryInfC{$\mathrm{show \big(read\ \langle 5\rangle\big)} : \forall \alpha.\ (\mathrm{Read}\ \alpha, \mathrm{Show}\ \alpha).\ \mathrm{String}$}
\end{prooftree}
\end{mdframed}
\caption{Inferencja niejednoznacznego typu}
\end{figure}

Wynikowy typ zawiera restrykcje odnoszące się do zmiennej $\alpha$ niewystępującej poza kontekstem, więc wyrażenie takie nie posiada prawidłowego typu\cite{Report}.
% Skoro jest jeden result to kolejka może się usuwać jeśli nikt jej nie słucha, doczytać
\chapter{Przykłady elementów pracy dyplomowej}

\section{Liczba}

Pakiet \texttt{siunitx} zadba o to, by liczba została poprawnie sformatowana: \\
\begin{center}
	\num{1234567890.0987654321}
\end{center}


\section{Rysunek}

Pakiet \texttt{subcaption} pozwala na umieszczanie w podpisie rysunku odnośników do ,,podilustracji'': \\

\begin{figure}[h]
	\centering
	\begin{subfigure}{0.35\textwidth}
		\centering
		\framebox[2.0\width]{A}
		\subcaption{\label{subfigure_a}}
	\end{subfigure}
	\begin{subfigure}{0.35\textwidth}
		\centering
		\framebox[2.0\width]{B}
		\subcaption{\label{subfigure_b}}
	\end{subfigure}
	
	\caption{\label{fig:subcaption_example}Przykład użycia \texttt{\textbackslash subcaption}: \protect\subref{subfigure_a} litera A, \protect\subref{subfigure_b} litera B.}
\end{figure}

\section{Tabela}

Pakiet \texttt{threeparttable} umożliwia dodanie do tabeli adnotacji: \\

\begin{table}[h]
	\centering
	
	\begin{threeparttable}
		\caption{Przykład tabeli}
		\label{tab:table_example}
		
		\begin{tabularx}{0.6\textwidth}{C{1}}
			\toprule
			\thead{Nagłówek\tnote{a}} \\
			\midrule
			Tekst 1 \\
			Tekst 2 \\
			\bottomrule
		\end{tabularx}
		
		\begin{tablenotes}
			\footnotesize
			\item[a] Jakiś komentarz\textellipsis
		\end{tablenotes}
		
	\end{threeparttable}
\end{table}

\section{Wzory matematyczne}

Czasem zachodzi potrzeba wytłumaczenia znaczenia symboli użytych w równaniu. Można to zrobić z użyciem zdefiniowanego na potrzeby niniejszej klasy środowiska \texttt{eqwhere}.

\begin{equation}
E = mc^2
\end{equation}
gdzie
\begin{eqwhere}[2cm]
	\item[$m$] masa
	\item[$c$] prędkość światła w próżni
\end{eqwhere}

Odległość półpauzy od lewego marginesu należy dobrać pod kątem najdłuższego symbolu (bądź listy symboli) poprzez odpowiednie ustawienie parametru tego środowiska (domyślnie: 2 cm).


% itd.
% \appendix
% \include{dodatekA}
% \include{dodatekB}
% itd.
\printbibliography

\end{document}
