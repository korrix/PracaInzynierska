\chapter{RabbitMQ}
\label{cha:rabbitmq}
RabbitMQ to otwartoźródłowy broker wiadomości - oprogramowanie zapewniające odporny na zakłócenia mechanizm komunikacji sieciowej. Został zaimplementowany w języku Erlang z wykorzystaniem Open Telecom Platform. Posiada biblioteki dla wszystkich znaczących języków programowania - w tym dla Haskella (biblioteka amqp - od ang. Advanced Message Queuing Protocol). Dwie najważniejsze koncepcje implementowane przez RabbitMQ to - \textbf{kolejki} (ang. queues) oraz \textbf{wymienniki} (ang. exchanges). Kolejki to przetrzymywane przez serwer struktury FIFO z sieciowym protokołem dostępu. Są najbardziej podstawową abstrakcją stosowaną w rozproszonym programowaniu współbieżnym, a w przypadku projektu będącego przedmiotem tej pracy pozwalają wielu węzłom ''konkurować'' o zadania w bezpieczny sposób (nigdy nie dojdzie do sytuacji, w której dwa zadania są w tym samym czasie wykonywane na dwóch różnych węzłach). Wymienniki są pośrednikami dostępu do zbiorów kolejek - wiadomość, która zostanie dodana do wymiennika może zostać powielona i rozdystrybuowana do połączonych z wymiennikiem kolejek w jeden z czterech możliwych sposobów zależnie od predefiniowanego typu wymiennika:
\begin{description}[align=left]
  \item [direct] wiadomość trafia tylko do kolejek z określonym kluczem routującym, identycznym z kluczem zawartym w nagłówku wiadomości
  \item [fanout] wiadomość trafia do wszystkich podłączonych kolejek
  \item [topic] wiadomość trafia do konkretnej kolejki tylko wtedy, kiedy jej klucz routujący spełnia wzorzec określony dla tej kolejki 
  \item [headers] wiadomość trafia do konkretnej kolejki na podstawie innych niż klucz routujący parametrów zawartych w nagłówku
  \end{description}