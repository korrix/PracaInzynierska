% !TEX root = praca.tex

\chapter{Programowanie rozproszone}
\label{cha:programowanieRozproszone}
Programowanie rozproszone to styl programowania umożliwiający [TODO]
\section{Architektura}
\label{sec:architektura}

\subsection{Klient-serwer}
\label{ssec:klientSerwer}
W tym podejściu oprogramowanie klienckie łączy się z serwerem celem pobrania danych wejściowych, a po ich przetworzeniu odsyła wyniki z powrotem

\subsection{Peer-to-peer}
\label{ssec:peer2peer}


\section{Modele obliczeń rozproszonych}
\label{sec:modeleObl}

\subsection{Zcentralizowany}
\label{ssec:zcentralizowany}
(jeden węzeł nadzoruje pracę innych)

\subsection{Równoległy}
\label{ssec:rownolegly}
(jeden węzeł wykonuje to samo zadanie na wielu komputerach)