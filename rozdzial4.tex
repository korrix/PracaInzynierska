% !TEX root = praca.tex
\chapter{Język Haskell}
\label{cha:haskell}
Haskell to funkcyjny język programowania, którego nazwa pochodzi od imienia amerykańskiego logika Haskella Curry'ego. Posiada silny, statyczny system typów z inferencją umożliwiający wychwytywanie wielu błędów oprogramowania już na etapie kompilacji. Jest głównie wykorzystywany w kręgach akademickich ze względu na implementację wielu eksperymentalnych rozwiązań takich jak np. Leniwe wartościowanie, reprezentowanie efektów ubocznych funkcji za pomocą monad czy statyczny, parametryczny polimorfizm. 

\section{Programowanie funkcyjne}
\label{sec:funprog}
Programowanie funkcyjne to paradygmat programowania, w którym podstawowym budulcem programów są funkcje i ich złożenia. Wynik działania programu jest ustalany poprzez wartościowanie złożonych wyrażeń funkcyjnych, a dokładna kolejność wykonywanych operacji nie jest specyfikowana w kodzie (najczęściej ustala ją kompilator języka). Istnieją dwie podstawowe kategorie funkcji - funkcje czyste oraz funkcje z efektami ubocznymi. Funkcja czysta dla ustalonych argumentów zawsze zwraca dokładnie tą samą wartość 

\section{Podstawowe cechy języka}
\label{sec:cechyjezyka}

\section{Opis zastosowanych bibliotek}
\label{sec:cechyjezyka}
\begin{description}[align=left]
  \item [amqp] Klient brokera RabbitMQ
  \item [bytestring] Wydajna implmentacja ciągów bajtów (niewykorzystująca list)
  \item [configurator] Parser plików konfiguracyjnch z prostym dostępem do wartości atrybutów na podstawie nazwy
  \item [store] Wydajna biblioteka do serializacji struktur danych
  \item [uuid] Biblioteka dostarczająca funkcje generowania uniwersalnie unikalnych identyfikatorów
  \item [stm] Programowa pamięć transakcyjna, umożliwiająca wygodną oraz bezpieczną synchronizację wątków
\end{description}