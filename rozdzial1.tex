% !TEX root = praca.tex

\chapter{Wprowadzenie}
\label{cha:wprowadzenie}
W czasach kiedy przewidywania Gordona Moore'a dotyczące dalszego wzrostu mocy obliczeniowej pojedynczych komputerów przestają się sprawdzać, coraz częściej wykorzystujemy metody wykonywania programów oparte na jednoczesnym przetwarzaniu rozproszonym na wielu połączonych ze sobą maszynach. 

Pomimo pozornej prostoty takiego rozwiązania istnieje bardzo niewiele narzędzi ułatwiających programowanie w modelu rozproszonym.

%---------------------------------------------------------------------------
\section{Cel i założenia pracy}
\label{sec:celePracy}
Celem poniższej pracy jest implementacja w języku Haskell podstawowej biblioteki do rozpraszania zadań na wielu komputerach z wykorzystaniem brokera RabbitMQ, umożliwiającą zlecanie zadań do wykonania, przerywanie zadań będących w trakcie wykonywania, raportowanie na bieżąco postępów wykonania, przesyłania wyników oraz wprowadzania zależności pomiędzy zadaniami.

%---------------------------------------------------------------------------
\section{Istniejące rozwiązania}
\label{sec:celePracy}
Poniższa lista zawiera skrócony opis istniejących bibliotek programistycznych, służących do obliczeń rozproszonych:
\subsection{Celery}
\label{ssec:celery}
\textbf{Celery} jest asynchroniczną kolejką zadań opartą o rozproszone komunikaty przesyłane między komputerami, napisaną w języku Python. Działa w czasie rzeczywistym, jednak umożliwia również szeregowanie zadań. Interfejs programistyczny pozwala na zlecanie zadań zarówno w sposób synchroniczny jak i asynchroniczny, oraz przesyłanie wyników. Wspiera wiele brokerów wiadomości (np. RabbitMQ, Redis, MongoDB). 

\subsection{Resque}
\label{ssec:resque}
\textbf{Resque} jest biblioteką języka Ruby, wykorzystującą bazę Redis w charakterze brokera wiadomości. Umożliwia asynchroniczne zlecanie powtarzalnych zadań na innych komputerach. Biblioteka ta jest często używana przez programistów serwisów internetowych do wykonywania długotrwałych operacji (np. generowanie miniaturek zdjęć, rozsyłanie newslettera e-mail, tworzenie raportów, etc...) 

\subsection{Cloud Haskell}
\label{ssec:cloudHs}
\textbf{Cloud Haskell} to biblioteka języka Haskell, udostępniająca warstwę transportową do komunikacji między węzłami, mechanizm serializacji domknięć pozwalający na zdalne uruchamianie procesów oraz API do programowania rozproszonego.
%---------------------------------------------------------------------------

















