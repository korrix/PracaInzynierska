% !TEX root = praca.tex

\chapter{Wprowadzenie}
\label{cha:wprowadzenie}
W czasach kiedy przewidywania Gordona Moore'a dotyczące dalszego wzrostu mocy obliczeniowej pojedynczych komputerów przestają się sprawdzać, coraz częściej wykorzystujemy metody wykonywania programów oparte na jednoczesnym przetwarzaniu rozproszonym na wielu połączonych ze sobą maszynach. 

Pomimo pozornej prostoty takiego rozwiązania istnieje bardzo niewiele narzędzi ułatwiających programowanie w modelu rozproszonym.

%---------------------------------------------------------------------------
\section{Cel i założenia pracy}
\label{sec:celePracy}
Celem poniższej pracy jest implementacja w języku Haskell podstawowej biblioteki do rozpraszania zadań na wielu komputerach z wykorzystaniem brokera RabbitMQ, umożliwiającą zlecanie zadań do wykonania, przerywanie zadań będących w trakcie wykonywania, raportowanie na bieżąco postępów wykonania, przesyłania wyników oraz wprowadzanie zależności pomiędzy zadaniami.

%---------------------------------------------------------------------------
\section{Zawartość pracy}
\label{sec:zawartoscPracy}
W rozdziale 1 cośtamcośtam...

















